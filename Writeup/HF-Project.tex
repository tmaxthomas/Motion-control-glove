\documentclass{article}
\begin{document}
	\section{Introduction}
	 The purpose of this project is to create a glove, designed to capture user hand motions and translate them to commands for a connected computer. Ostensibly, this can be applied to any number of use cases, but the intended application is the improvement of immersion in computer games.
	 
	 Currently, there are a number of motion-control technologies available for this purpose. However, each of these comes with a number of disadvantages. Motion controllers, such as those for the Nintendo Wii, require button pressed combined with motion of the controller itself. The amount of motion often required can also be detrimental, as it has the potential to quickly leave the player worn out. There are also dedicated virtual reality systems, such as the HTC Vive. This provides a superior visual and auditory environment, but still fail to address the issue of incongruous button presses. Furthermore, such virtual reality systems require a large, unobstructed area in order to function properly. There is also a larger issue that none of these systems address-they all require that the games they are being used for come with support programmed in; support which is unusual in recent titles and entirely lacking in older ones. This project seeks to address all of these issues.
	 
	 The issue of button presses is easy to address-the glove doesn't have any buttons-it relies entirely on user movements. The issue of motion is also relatively easy-small motions of the hand, of the sort the glove is designed to detect, do not require much in the way of physical effort from the user. These small motions also address the issue of space required-the user does not require any more space than they would need to use their computer in the first place. The final issue, that of support, is addressed by the glove's reliance on the user to define the set of movement-to-keystroke relations (henceforth to be referred to as "macros") they wish to use. Thus, the user can personalize the glove to their taste, working with whatever is most comfortable for them.
	 
	 It is worth noting that this product is not designed for commercial release. It is a proof-of-concept and an interesting gadget designed to address an idle thought of one person. It is unlikely it could be a commercial success, due to limitations of the technology and the concept itself. It was made because it seemed interesting, not because it seemed like a viable product.
\end{document}